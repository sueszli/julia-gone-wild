\documentclass[a4paper,%
8pt, % font size
DIV99, % 1.00 text width
headsepline, % line under header
headings=normal,
]{scrartcl}

\usepackage{./docs/bopc}
\usepackage{csvsimple}
\usepackage{listings}
\usepackage{booktabs}
\usepackage{array}


%% ==================== ADAPT ACCORDINGLY ==================== %%

% args: first name, last name, matriculation number
\setAuthorOne{Pia}{Schwarzinger}{12017370}
\setAuthorTwo{Yahya}{Jabary}{11912007}

% set group size and group number, group size influences author formatting

% args: group size, group number (TUWEL)
\setGroup{2}{13}

%% =========================================================== %%

\begin{document}

\maketitlepage

%% ================= SUBMISSION INFORMATION ================== %%

% Please fill out accordingly

%% ========================= REPORT ========================== %%

\section{Exercise 1}

\lstinputlisting{./demo/scheduler.c}

\subsection{What do \texttt{a} and \texttt{t} count?}

The variable \texttt{a} stores the selected thread number for each parallel iteration, while \texttt{t} stores a non-atomic counter that all threads with the same ID increment. Unless no two threads are assigned the same iteration, the final value of \texttt{t} will be non-deterministic as each \texttt{var++} operation is in fact a read-modify-write operation:

\begin{lstlisting}
movl  -4(%rbp), %eax   # load var into eax
addl  $1, %eax         # increment eax by 1
movl  %eax, -4(%rbp)   # store eax back into var
\end{lstlisting}

\subsection{Values for all elements in \texttt{a} and \texttt{t}}

\begin{table}[htbp]
    \centering
    \caption{Values of array \texttt{a} for different scheduling strategies}
    \begin{tabular}{l*{17}{c}}
        \toprule
        case / \texttt{a} & 0 & 1 & 2 & 3 & 4 & 5 & 6 & 7 & 8 & 9 & 10 & 11 & 12 & 13 & 14 & 15 & 16 \\
        \midrule
        \texttt{static, 0} & 0 & 0 & 0 & 0 & 0 & 0 & 1 & 1 & 1 & 1 & 1 & 1 & 2 & 2 & 2 & 2 & 2 \\
        \texttt{static, 1} & 0 & 1 & 2 & 0 & 1 & 2 & 0 & 1 & 2 & 0 & 1 & 2 & 0 & 1 & 2 & 0 & 1 \\
        \texttt{dynamic, 1} & 1 & 1 & 1 & 1 & 1 & 1 & 1 & 1 & 1 & 1 & 1 & 1 & 1 & 1 & 1 & 1 & 1 \\
        \texttt{dynamic, 2} & 0 & 0 & 0 & 0 & 0 & 0 & 0 & 0 & 0 & 0 & 0 & 0 & 0 & 0 & 0 & 0 & 0 \\
        \texttt{guided, 5} & 0 & 0 & 0 & 0 & 0 & 0 & 0 & 0 & 0 & 0 & 0 & 0 & 0 & 0 & 0 & 0 & 0 \\
        \bottomrule
    \end{tabular}
\end{table}

\begin{table}[htbp]
    \centering
    \caption{Values of array \texttt{t} for different scheduling strategies - keep in mind that these values are not reproducible / deterministic.}
    \begin{tabular}{l*{3}{c}}
        \toprule
        case / \texttt{t} & 0 & 1 & 2 \\
        \midrule
        \texttt{static, 0} & 74307862 & 7 & 1806905557 \\
        \texttt{static, 1} & 8591638 & 7 & 1872621781 \\
        \texttt{dynamic, 1} & 6150416 & 18 & 1875062992 \\
        \texttt{dynamic, 2} & 40737057 & 1 & 1840476368 \\
        \texttt{guided, 5} & 51370273 & 1 & 1829843168 \\
        \bottomrule
    \end{tabular}
\end{table}

\section{Exercise 2}

\subsection{Optimal Schedule}

\begin{table}[htbp]
    \centering
    \caption{Duration of independent tasks we want to schedule optimally.}
    \begin{tabular}{l*{17}{c}}
        \toprule
        Task ID & 0 & 1 & 2 & 3 & 4 & 5 & 6 & 7 & 8 & 9 & 10 & 11 & 12 & 13 & 14 & 15 & 16 \\
        \midrule
        Task duration & 1 & 2 & 1 & 2 & 1 & 2 & 1 & 2 & 1 & 2 & 1 & 2 & 1 & 2 & 4 & 3 & 3 \\
        \bottomrule
    \end{tabular}
\end{table}


\subsection{Schedule \texttt{static,3}}

\subsection{Schedule \texttt{dynamic,2}}

\section{Exercise 3}

\subsection{Fix the problems with this OpenMP code}

\section{Exercise 4}

\subsection{What is the output of the three different versions?}

\subsection{How often is the function \texttt{omp\_tasks} called?}

\section{Exercise 5}

\subsection{Parallelize the pixel computation}

\subsection{Running time analysis}

\subsection{Influence of schedule parameter}

\section{Exercise 6}

\subsection{Parallelize the filter computation}

\subsection{Strong scaling analysis}

\subsection{Weak scaling analysis}

\section{Exercise 7}

\subsection{Convert OpenMP code to CUDA}

\subsection{Running time analysis}

\subsection{Impact of block size}

\subsection{Running time: CPU vs GPU code}

\newpage

\section{Addendum: Raw Data}

\begin{figure}[htbp]
    \centering
    \csvautotabular{./output/filter_strong.csv}\label{tab:filter_strong}
    \caption{Raw output from "filter strong" job.}
\end{figure}

\begin{figure}[htbp]
    \centering
    \csvautotabular{./output/filter_weak.csv}\label{tab:filter_weak}
    \caption{Raw output from "weak scaling" job. Timed out on \textit{slurmstepd} due to time out / time limit.}
\end{figure}

\begin{figure}[htbp]
    \centering
    \csvautotabular{./output/juliap.csv}\label{tab:juliap}
    \caption{Raw output from "juliap" job.}
\end{figure}

\begin{figure}[htbp]
    \centering
    \csvautotabular{./output/juliap2.csv}\label{tab:juliap2}
    \caption{Raw output from "juliap2" job.}
\end{figure}

\end{document}
